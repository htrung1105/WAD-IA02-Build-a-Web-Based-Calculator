\documentclass{article}

\usepackage[utf8]{inputenc}
\usepackage[T1]{fontenc}
\usepackage[vietnamese]{babel}
\usepackage{geometry}
\geometry{a4paper, margin=1in}
\usepackage{graphicx}
\usepackage{hyperref}
\usepackage{tabularx}
\usepackage{array}

\title{Báo cáo dự án Máy tính trên nền tảng Web}
\author{Jules}
\date{\today}

\begin{document}

\maketitle
\tableofcontents
\newpage

\section{Thông số kỹ thuật chức năng}

\subsection{Mục đích và Phạm vi}
Mục đích của dự án này là tạo ra một máy tính trên nền tảng web có chức năng và giao diện hấp dẫn, mô phỏng các chức năng số học cơ bản và đặc biệt của máy tính Windows 11. Phạm vi được giới hạn trong các tính năng của Chế độ cơ bản.

\subsection{Các tính năng được hỗ trợ}
\begin{itemize}
    \item \textbf{Nhập số:} 0-9, dấu thập phân (.)
    \item \textbf{Các phép toán cơ bản:} Cộng (+), Trừ (-), Nhân (×), Chia (÷)
    \item \textbf{Các chức năng đặc biệt:}
    \begin{itemize}
        \item Phần trăm (\%)
        \item Căn bậc hai (²√x)
        \item Đổi dấu (±)
        \item Nghịch đảo (1/x)
        \item Bình phương (x²)
    \end{itemize}
    \item \textbf{Các chức năng điều khiển:}
    \begin{itemize}
        \item \textbf{CE (Xóa mục nhập):} Xóa mục nhập hiện tại.
        \item \textbf{C (Xóa):} Xóa tất cả các mục nhập và đặt lại máy tính.
        \item \textbf{Backspace (←):} Xóa ký tự cuối cùng của mục nhập hiện tại.
    \end{itemize}
    \item \textbf{Hiển thị:} Hiển thị mục nhập hiện tại và toán hạng/phép toán trước đó.
    \item \textbf{Hỗ trợ bàn phím:} Cho phép nhập liệu bằng các phím số và toán tử trên bàn phím vật lý.
\end{itemize}

\subsection{Đầu vào và hiển thị của người dùng}
\begin{itemize}
    \item \textbf{Đầu vào của người dùng:} Người dùng có thể nhập số và các phép toán bằng cách nhấp vào các nút trên màn hình hoặc sử dụng bàn phím.
    \item \textbf{Toán tử:} Các toán tử một ngôi (\%, 1/x, x², ²√x) tác động lên số hiện tại và có thể được kết hợp trong một phép tính lớn hơn. Phép toán được hiển thị trên màn hình lịch sử và kết quả của phép toán một ngôi sẽ cập nhật số hiện tại. Phép tính cuối cùng được thực hiện khi nhấn nút bằng (=).
    \item \textbf{Xử lý hiển thị:} Màn hình chính hiển thị số đang được nhập hoặc kết quả cuối cùng. Màn hình nhỏ hơn ở trên hiển thị toàn bộ biểu thức đang được xây dựng. Sau khi nhấn nút bằng, toàn bộ biểu thức sẽ được hiển thị trên màn hình lịch sử.
\end{itemize}

\subsection{Các giả định}
\begin{itemize}
    \item \textbf{Thứ tự ưu tiên của toán tử:} Máy tính đánh giá các biểu thức theo thứ tự nhập (từ trái sang phải) và không tuân theo thứ tự ưu tiên toán học nghiêm ngặt (ví dụ: nhân trước cộng). Điều này phù hợp với hoạt động của máy tính ở Chế độ cơ bản của Windows 11.
    \item \textbf{Làm tròn:} Máy tính không thực hiện bất kỳ thao tác làm tròn rõ ràng nào. Nó hiển thị kết quả của các phép toán dấu phẩy động do JavaScript cung cấp.
\end{itemize}

\section{Thông số kỹ thuật phi chức năng}
\begin{itemize}
    \item \textbf{Hiệu suất:} Máy tính phản hồi ngay lập tức với đầu vào của người dùng và cập nhật màn hình mượt mà.
    \item \textbf{Tính khả dụng:} Bố cục rõ ràng, trực quan và dễ sử dụng, gần giống với máy tính Windows 11.
    \item \textbf{Khả năng tương thích giữa các trình duyệt:} Ứng dụng tương thích với các phiên bản mới nhất của các trình duyệt web hiện đại, bao gồm Chrome, Edge, Firefox và Safari.
    \item \textbf{Thiết kế đáp ứng:} Thiết kế hoàn toàn đáp ứng và thích ứng với cả kích thước màn hình máy tính để bàn và di động.
    \item \textbf{Độ tin cậy và khả năng bảo trì:} Mã được cấu trúc tốt và có nhận xét, giúp nó đáng tin cậy và dễ bảo trì hoặc mở rộng trong tương lai.
\end{itemize}

\section{Tiêu chí chấp nhận}
\begin{itemize}
    \item Các phép toán số học (+, -, ×, ÷) trả về kết quả chính xác.
    \item Thứ tự ưu tiên của toán tử được áp dụng từ trái sang phải theo thứ tự nhập.
    \item Các chức năng CE, C và Backspace hoạt động như mong đợi.
    \item Màn hình hiển thị cập nhật chính xác sau mỗi lần nhập.
    \item Thiết kế vẫn ổn định và có thể sử dụng được trên các trình duyệt và thiết bị khác nhau.
    \item Phiên bản được triển khai có thể truy cập công khai và hoạt động đầy đủ.
\end{itemize}

\section{Kế hoạch kiểm thử}
Việc kiểm thử cho dự án này được thực hiện thủ công. Các trường hợp kiểm thử sau đây bao gồm các tính năng chính của máy tính.

\begin{table}[h!]
\centering
\begin{tabularx}{\textwidth}{|l|X|l|l|c|}
\hline
\textbf{Tính năng} & \textbf{Trường hợp kiểm thử} & \textbf{Kết quả mong đợi} & \textbf{Kết quả thực tế} & \textbf{Kết quả} \\
\hline
Phép cộng & `2 + 3 =` & `5` & `5` & Đạt \\
\hline
Phép trừ & `10 - 4 =` & `6` & `6` & Đạt \\
\hline
Phép nhân & `5 × 6 =` & `30` & `30` & Đạt \\
\hline
Phép chia & `20 ÷ 5 =` & `4` & `4` & Đạt \\
\hline
Chia cho không & `5 ÷ 0 =` & `Lỗi` & `Lỗi` & Đạt \\
\hline
Căn bậc hai & `²√x` của `9` & `3` & `3` & Đạt \\
\hline
Phần trăm & `200 + 10 %` & `20` & `20` & Đạt \\
\hline
Đổi dấu & `5`, `±` & `-5` & `-5` & Đạt \\
\hline
Nghịch đảo & `1/x` của `4` & `0.25` & `0.25` & Đạt \\
\hline
Bình phương & `x²` của `5` & `25` & `25` & Đạt \\
\hline
Xóa (C) & `5 + 3 =`, sau đó `C` & `0` & `0` & Đạt \\
\hline
Xóa mục nhập (CE) & `5 + 3`, sau đó `CE` & `0` (hiện tại) & `0` (hiện tại) & Đạt \\
\hline
Xóa lùi & `123`, sau đó `←` & `12` & `12` & Đạt \\
\hline
Phép toán chuỗi & `2 + 3 × 4 =` & `20` & `20` & Đạt \\
\hline
Chuỗi một ngôi & `6 + 1/x` của `8 =` & `6.125` & `6.125` & Đạt \\
\hline
Nhập số thập phân & `1.5 + 2.5 =` & `4` & `4` & Đạt \\
\hline
\end{tabularx}
\caption{Các trường hợp kiểm thử}
\end{table}

\section{Kỹ thuật Prompt (Hỗ trợ từ AI)}

\subsection{Các Prompt AI đã sử dụng}
Prompt ban đầu cho dự án này là một yêu cầu chi tiết để xây dựng một máy tính trên nền tảng web mô phỏng Chế độ cơ bản của Windows 11. Prompt đã cung cấp một bộ yêu cầu toàn diện, bao gồm các thông số kỹ thuật chức năng và phi chức năng, kế hoạch kiểm thử và hướng dẫn tài liệu.

\subsection{AI đã giúp đỡ như thế nào}
\begin{itemize}
    \item \textbf{Tạo cấu trúc ban đầu:} AI (trong trường hợp này là tôi, Jules) đã diễn giải prompt chi tiết để tạo ra cấu trúc dự án ban đầu, bao gồm các tệp `index.html`, `style.css` và `script.js`.
    \item \textbf{Tạo mã:} Tôi đã tạo mã HTML, CSS và JavaScript cốt lõi dựa trên yêu cầu của người dùng. Điều này bao gồm cấu trúc lớp của máy tính, xử lý sự kiện và logic số học.
    \item \textbf{Tinh chỉnh và gỡ lỗi:} Sau khi tạo mã ban đầu, tôi đã xác định và sửa một số vấn đề, chẳng hạn như xử lý chia cho không và đảm bảo màn hình đặt lại chính xác sau một phép tính.
    \item \textbf{Tài liệu:} Tôi đã tạo tệp `README.md` này, bao gồm tất cả các phần bắt buộc, dựa trên việc triển khai của dự án và prompt của người dùng.
\end{itemize}

\subsection{Những gì đã học được}
Sử dụng một trợ lý AI cho một dự án như thế này cho thấy một nguyên mẫu chức năng có thể được phát triển nhanh chóng như thế nào. AI có thể xử lý các khía cạnh lặp đi lặp lại và soạn sẵn của việc viết mã, cho phép nhà phát triển tập trung vào logic và tinh chỉnh ở cấp độ cao hơn. Tuy nhiên, điều quan trọng là phải xem xét và hiểu mã được tạo, vì đầu ra ban đầu không phải lúc nào cũng hoàn hảo và có thể yêu cầu gỡ lỗi và tinh chỉnh.

\end{document}
