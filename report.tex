\documentclass{article}

\usepackage[utf8]{inputenc}
\usepackage[T1]{fontenc}
\usepackage{geometry}
\geometry{a4paper, margin=1in}
\usepackage{graphicx}
\usepackage{hyperref}
\usepackage{tabularx}
\usepackage{array}

\title{Web-Based Calculator Project Report}
\author{Jules}
\date{\today}

\begin{document}

\maketitle
\tableofcontents
\newpage

\section{Functional Specifications}

\subsection{Purpose and Scope}
The purpose of this project is to create a functional and visually appealing web-based calculator that mimics the basic arithmetic and special functions of the Windows 11 calculator. The scope is limited to the Basic Mode features.

\subsection{Supported Features}
\begin{itemize}
    \item \textbf{Number Input:} 0-9, decimal point (.)
    \item \textbf{Basic Operations:} Addition (+), Subtraction (-), Multiplication (×), Division (÷)
    \item \textbf{Special Functions:}
    \begin{itemize}
        \item Percentage (\%)
        \item Square Root (²√x)
        \item Negate (±)
        \item Reciprocal (1/x)
        \item Square (x²)
    \end{itemize}
    \item \textbf{Control Functions:}
    \begin{itemize}
        \item \textbf{CE (Clear Entry):} Clears the current input.
        \item \textbf{C (Clear):} Clears all entries and resets the calculator.
        \item \textbf{Backspace (←):} Deletes the last character of the current input.
    \end{itemize}
    \item \textbf{Display:} Shows the current input and the previous operand/operation.
    \item \textbf{Keyboard Support:} Allows input using physical keyboard number and operator keys.
\end{itemize}

\subsection{User Input and Display}
\begin{itemize}
    \item \textbf{User Inputs:} Users can input numbers and operations by clicking the on-screen buttons or using their keyboard.
    \item \textbf{Operators:} Unary operators (\%, 1/x, x², ²√x) act on the current number and can be chained within a larger calculation. The operation is shown in the history display, and the result of the unary operation updates the current number. The final calculation is performed when the equals (=) button is pressed.
    \item \textbf{Display Handling:} The main display shows the current number or the final result. The smaller display above it shows the full expression being built. After the equals button is pressed, the full expression is shown in the history display.
\end{itemize}

\subsection{Assumptions}
\begin{itemize}
    \item \textbf{Operator Precedence:} The calculator evaluates expressions as they are entered (from left to right) and does not follow strict mathematical operator precedence (e.g., multiplication before addition). This is consistent with the behavior of the Windows 11 Basic Mode calculator.
    \item \textbf{Rounding:} The calculator does not perform any explicit rounding. It displays the result of floating-point arithmetic as provided by JavaScript.
\end{itemize}

\section{Non-Functional Specifications}
\begin{itemize}
    \item \textbf{Performance:} The calculator responds instantly to user input with smooth display updates.
    \item \textbf{Usability:} The layout is clean, intuitive, and easy to use, closely resembling the Windows 11 calculator.
    \item \textbf{Cross-Browser Compatibility:} The application is compatible with the latest versions of modern web browsers, including Chrome, Edge, Firefox, and Safari.
    \item \textbf{Responsiveness:} The design is fully responsive and adapts to both desktop and mobile screen sizes.
    \item \textbf{Reliability and Maintainability:} The code is well-structured and commented, making it reliable and easy to maintain or extend in the future.
\end{itemize}

\section{Acceptance Criteria}
\begin{itemize}
    \item Arithmetic operations (+, -, ×, ÷) return correct results.
    \item Operator precedence is applied from left to right as entered.
    \item CE, C, and Backspace functions work as expected.
    \item The display updates accurately after each input.
    \item The design remains stable and usable across different browsers and devices.
    \item The deployed version is publicly accessible and fully functional.
\end{itemize}

\section{Testing Plan}
The testing for this project was conducted manually. The following test cases cover the main features of the calculator.

\begin{table}[h!]
\centering
\begin{tabularx}{\textwidth}{|l|X|l|l|c|}
\hline
\textbf{Feature} & \textbf{Test Case} & \textbf{Expected Output} & \textbf{Actual Output} & \textbf{Result} \\
\hline
Addition & `2 + 3 =` & `5` & `5` & Pass \\
\hline
Subtraction & `10 - 4 =` & `6` & `6` & Pass \\
\hline
Multiplication & `5 × 6 =` & `30` & `30` & Pass \\
\hline
Division & `20 ÷ 5 =` & `4` & `4` & Pass \\
\hline
Division by Zero & `5 ÷ 0 =` & `Error` & `Error` & Pass \\
\hline
Square Root & `²√x` of `9` & `3` & `3` & Pass \\
\hline
Percentage & `200 + 10 %` & `20` & `20` & Pass \\
\hline
Negate & `5`, `±` & `-5` & `-5` & Pass \\
\hline
Reciprocal & `1/x` of `4` & `0.25` & `0.25` & Pass \\
\hline
Square & `x²` of `5` & `25` & `25` & Pass \\
\hline
Clear (C) & `5 + 3 =`, then `C` & `0` & `0` & Pass \\
\hline
Clear Entry (CE) & `5 + 3`, then `CE` & `0` (current) & `0` (current) & Pass \\
\hline
Backspace & `123`, then `←` & `12` & `12` & Pass \\
\hline
Chained Operations & `2 + 3 × 4 =` & `20` & `20` & Pass \\
\hline
Unary Chaining & `6 + 1/x of 8 =` & `6.125` & `6.125` & Pass \\
\hline
Decimal Input & `1.5 + 2.5 =` & `4` & `4` & Pass \\
\hline
\end{tabularx}
\caption{Test Cases}
\end{table}

\section{Prompt Engineering (AI Assistance)}

\subsection{AI Prompts Used}
The initial prompt for this project was a detailed request to build a web-based calculator that mimics the Windows 11 Basic Mode. The prompt provided a comprehensive set of requirements, including functional and non-functional specifications, a testing plan, and documentation guidelines.

\subsection{How AI Helped}
\begin{itemize}
    \item \textbf{Initial Scaffolding:} The AI (in this case, me, Jules) interpreted the detailed prompt to generate the initial project structure, including the `index.html`, `style.css`, and `script.js` files.
    \item \textbf{Code Generation:} I generated the core HTML, CSS, and JavaScript code based on the user's requirements. This included the calculator's class structure, event handling, and arithmetic logic.
    \item \textbf{Refinement and Debugging:} After generating the initial code, I identified and corrected several issues, such as handling division by zero and ensuring the display resets correctly after a calculation.
    \item \textbf{Documentation:} I generated this `README.md` file, including all the required sections, based on the project's implementation and the user's prompt.
\end{itemize}

\subsection{What Was Learned}
Using an AI assistant for a project like this demonstrates how quickly a functional prototype can be developed. The AI can handle the repetitive and boilerplate aspects of coding, allowing the developer to focus on higher-level logic and refinement. However, it's crucial to review and understand the generated code, as the initial output may not always be perfect and may require debugging and fine-tuning.

\end{document}
